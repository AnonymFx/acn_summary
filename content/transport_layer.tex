%!TEX root = ../report.tex

\section{Internet Transport Layer}
\subsubsection*{Congestion Control}
One of the transports layers jobs is to handle congestion.
Congestions happens if e.g.\ too many sources send too much data to fast for the network to handle which results in packet loss and long packet delays.\\
Congestion control tries to solve this because without controlling the outgoing traffic, capacity may drop dramatically because of congestion collapse.
\textbf{End-end congestion control} infers congestion only by observing lost packets and delay whereas \textbf{network-assisted congestion control} use informations of routers to detect congestion.
Also an explicit rate can be told to the sender by the network with the second approach.\\

\textbf{Self clocking congestion control} is another approach which sends a new packet for every packet that left the network what the sender knows from ACK messages.
It assumes though that packet loss only occurs due to congestion which is not true for wireless networks for example.
