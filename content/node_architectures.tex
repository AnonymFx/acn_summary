%!TEX root = ../report.tex

\section{Node Architectures}
Nodes/Routers in a network consist of multiple input and output ports and a switching fabric in between.
There are three types of switching fabric: memory, a bus, or a crossbar (net of buses).\\
Usually the goal of a router is to completely process the incoming packets at line speed, but queuing/buffering might be necessary due to a too slow switching fabric or a overloading of one or multiple output ports.
Recommended buffer sizes are $RTT \cdot \text{link capacity}$ by RFC 3439 or $\frac{RTT \cdot \text{link capacity}} {\sqrt{\text{number of flows}}}$ more recently.\\

\textbf{First generation IP routers} had a CPU next to a global memory buffer which was connected to multiple line interfaces.
\textbf{The second generation} then introduced local memory buffers for all interfaces and the \textbf{third} moved the CPU to a separate card besides the network interfaces.
In \textbf{the forth generation} clustered and multistage network interfaces were added.
